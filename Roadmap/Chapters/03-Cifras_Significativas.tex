\chapter{Cifras Significativas}

\section{Definición}

Las \textbf{cifras significativas} representan una cantidad que tiene un origen real, usualmente obtenida mediante un instrumento de medición.

\subsection{Ejemplos}
\begin{enumerate}
    \item $34,156$ tiene \textbf{5 cifras significativas}.
    \item $0,00034156$ tiene \textbf{5 cifras significativas}. Los ceros iniciales no son significativos, ya que indican el orden de magnitud respecto a la unidad que acompaña.
    
    Ejemplo: Una longitud de $4 \, \text{cm}$ se expresa como $0,04 \, \text{m}$.
    
    \item $21,4000$ tiene \textbf{6 cifras significativas}. Los ceros finales son significativos porque indican el grado de exactitud de la cantidad.
    \item $28,000$ tiene \textbf{2 cifras significativas}. En un número entero, los últimos ceros sólo son significativos si representan el orden de magnitud respecto a la unidad que acompaña.
    \item $2054,01$ tiene \textbf{6 cifras significativas}
    
    \end{enumerate}

\newpage
\section{Redondeo de Cifras Significativas}

\subsection{Reglas de Redondeo}
\textbf{1.} Si la cifra que se omite es menor que $5$, entonces la cifra anterior se conserva sin cambios.\\  
   \textbf{Ejemplo:}  
   \begin{itemize}
       \item $24,782$ tiene \textbf{5 cifras significativas}.
       \item Al redondear a 4 cifras significativas: $24,782 \to 24,78$.
   \end{itemize}
\textbf{2.} Si la cifra que se omite es mayor o igual a $5$, entonces la cifra anterior aumenta en una unidad.\\  
   \textbf{Ejemplo:}  
   \begin{itemize}
       \item $24,796$ redondeado a \textbf{4 cifras significativas}:  
       $24,796 \to 24,80$.
       \item $4547$ redondeado a \textbf{3 cifras significativas}:  
       $4547 \to 4550$.
   \end{itemize}
Si la cifra que se omite es igual a $5$, hay dos posibles casos:
\subsection{Caso 1}
Entre las cifras suprimidas, además del $5$, hay otras distintas de cero.  
En este caso, la cifra anterior aumenta en una unidad.\\  
\textbf{Ejemplo:}  
\begin{itemize}
    \item $24,78502$ (7 cifras significativas).  
    Deseamos redondear a 4 cifras significativas:  
    $24,78502 \to 24,79$.
    \item $1/3 \to 0,33$.
\end{itemize}
\subsection{Caso 2}
Todas las cifras que se omiten son iguales a cero.
\begin{itemize}
    \item Si la cifra anterior es impar, aumenta en una unidad.
    \item Si la cifra anterior es par, no varía.
\end{itemize}

\textbf{Ejemplo: Redondear a 4 cifras significativas:}  
\begin{itemize}
    \item (a) $24,76500 \to 24,76$.
    \item (b) $24,77500 \to 24,78$.
    \item (c) $2477500 \to 2478000$.
\end{itemize}







