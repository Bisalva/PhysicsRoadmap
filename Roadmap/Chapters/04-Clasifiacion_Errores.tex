\chapter{Clasificación de Errores}

\section{Definición de error}
El \textbf{error o incertidumbre} es la diferencia entre el valor medido y el valor verdadero o real de la cantidad física.

Las cantidades se obtienen mediante un proceso de medición.

\subsection{Medida Directa}
\noindent
\textbf{Medida directa} $\longrightarrow$ Se obtienen directamente del instrumento. \\
\textbf{Ejemplos:} longitud, masa, trayectoria.

\subsection{Medida Indirecta}
\noindent
\textbf{Medida indirecta} $\longrightarrow$ Se necesita realizar algún cálculo a partir de medidas directas. \\
\textbf{Ejemplos:} área, Velocidad, etc.

\subsection{Consideraciones}
- Medir es comparar una cantidad con otra de la misma naturaleza, tomada como patrón.\\
- El valor verdadero (real) es imposible obtenerlo en una medición; cada proceso de medición viene acompañado de errores e incertidumbres.

\newpage
\section{Errores Sistemáticos}

Los errores sistemáticos permanecen durante todo el proceso de medición. \textbf{Afectan a todas las mediciones}.

\subsection{a) Instrumentales}
Son aquellos debidos a los aparatos de medición.  
Ejemplo: mala calibración o desgaste del instrumento.

\subsection{b) Personales}
Se originan debido a limitaciones humanas, como la vista, la audición o el tiempo de reacción.

\subsection{c) Método}
Surgen debido a que hay métodos que son más favorables que otros.

\section{Errores Accidentales}

Estos errores se producen en condiciones variables durante el proceso de medición.  
\textbf{ Dato: No afectan a todas las mediciones}.


\section{Exactitud, Precisión y Sensibilidad}

\subsection{Exactitud}
La exactitud es el grado de concordancia entre el valor medido y el valor verdadero.

\subsection{Precisión}
La precisión es el grado de concordancia entre una medida y otras de la misma cantidad física, realizadas bajo las mismas condiciones.

\subsection{Sensibilidad}
La sensibilidad de un instrumento es el valor mínimo de la cantidad física que puede medir. Para el instrumento, esto corresponde a la menor división posible.  
Ejemplo: Regla milimetrada, sensibilidad = 1 mm.

\newpage
\section{Error Absoluto \& Relativo}

\subsection{Error Absoluto}
El error absoluto se define como:
\[
\Delta X = |X_0 - X|
\]
Donde:
\begin{itemize}
    \item \( X_0 \): Valor verdadero (o real).
    \item \( X \): Valor medido.
\end{itemize}

\subsection{Error Relativo}
El error relativo se calcula como:
\[
\varepsilon = \frac{\Delta X}{|X|}
\]
Donde:
\begin{itemize}
    \item \( \Delta X \): Error absoluto.
    \item \( X \): Valor medido.
\end{itemize}

\subsection{Error Relativo Porcentual}
El error relativo porcentual se expresa como:
\[
\varepsilon (\%) = \varepsilon \cdot 100 = \frac{\Delta X}{|X|} \cdot 100
\]


\section {Presentación de una Medida}

Cuando se entrega formalmente una medida, se expresa de la siguiente manera:

\[
X \pm \Delta X
\]

Donde:
\begin{itemize}
    \item \( X \) es el \textbf{valor medido}.
    \item \( \Delta X \) es el \textbf{error absoluto}.
\end{itemize}

\noindent Esto indica que el valor verdadero \( X_0 \) se encuentra en algún punto dentro del intervalo:

\[
X - \Delta X \leq X_0 \leq X + \Delta X
\]

\section{Reglas en la Presentación de la Medida}

\subsection{Regla 1: Igual número de decimales}
El valor medido y el error deben tener el mismo número de decimales. 

Ejemplos:

\[
25.53 \pm 0.5 \quad \text{(Incorrecto)} \quad \Rightarrow \quad 25.5 \pm 0.5 \quad \text{(Correcto)}
\]

\[
25.53 \pm 0.005 \quad \text{(Incorrecto)} \quad \Rightarrow \quad 25.530 \pm 0.005 \quad \text{(Correcto)}
\]

\subsection{Regla 2: Orden de magnitud del error}
Cuando el valor medido y el error son números enteros, el orden de magnitud de la última cifra del valor medido debe ser igual o mayor al orden de magnitud del error.

Ejemplo:

\[
853 \pm 20 \text{ g} \quad \Rightarrow \quad 850 \pm 20 \text{ g}
\]

\subsection{Regla 3: Redondeo del error}
Para este curso, el error se debe entregar con una cifra significativa redondeada.

Ejemplo:

\[
\text{Cierta medida de masa: } \quad 25.12 \pm 0.05 \text{ g}
\]

\newpage
\section{Cálculo del Error Relativo Porcentual}

Dada la medida:

\[
25.12 \pm 0.05 \text{ g}
\]

Se desea calcular el error relativo porcentual.

\subsection{Formula}

El error relativo porcentual se define como:

\[
E (\%) = \frac{\Delta X}{X} \times 100
\]

Sustituyendo los valores:

\[
E (\%) = \frac{0.05}{25.12} \times 100 = 0.2\%
\]

\section{Cálculo del Error Absoluto en una Medida de Volumen}

Se mide el volumen de cierto líquido y se obtiene:

\[
X = 54 \text{ mL}
\]

con un error porcentual del \( 1\% \). Se desea encontrar el error absoluto \( \Delta X \) y expresar la medida en la forma:

\[
X \pm \Delta X
\]

\subsection{Formula}

La relación entre error absoluto y error porcentual es:

\[
E (\%) = \frac{\Delta X}{X} \times 100
\]

Despejando \( \Delta X \):

\[
1 = \frac{\Delta X \times 100}{54}
\]

\[
\Delta X = \frac{54}{100} = 0.54 \text{ mL}
\]

Por lo tanto, la medida expresada correctamente es:

\[
54 \pm 0.54 \text{ mL} \approx 54.0 \pm 0.5 \text{ mL}
\]

\section{Definición del Error Absoluto}

En general, la medida se expresa como:

\[
X \pm \Delta X
\]

donde el \textbf{error absoluto} se define como:

\[
\Delta X = |X_0 - X|
\]

siendo \( X_0 \) el valor verdadero o estimado.

\section{Factores que Afectan la Medición}

Existen dos factores principales que afectan la medición:
\begin{enumerate}
    \item Sensibilidad del instrumento.
    \item Estadística.
\end{enumerate}

\newpage
\section{Estimación del Error}
\subsection{Medidas Directas}

\subsubsection{Caso 1: Una única medida}

Si se realiza una única medida, el error se estima a partir de la sensibilidad del instrumento:

\begin{itemize}
    \item Si el instrumento es \textbf{analógico}: 
    \[
    \Delta X = \frac{S}{2}
    \]
    \item Si el instrumento es \textbf{digital}: 
    \[
    \Delta X = S
    \]
\end{itemize}

\subsubsection{Caso 2: Varias medidas}

Si se realizan varias mediciones, se deben calcular los siguientes valores:

\begin{itemize}
    \item \textbf{Valor medido:} Se obtiene como el promedio de los valores registrados:
    \[
    \bar{X} = \frac{1}{N} \sum_{i=1}^{N} X_i
    \]
    \item \textbf{Error:} Se estima utilizando la desviación estándar:
    \[
    \sigma = \sqrt{\frac{1}{N-1} \sum_{i=1}^{N} (X_i - \bar{X})^2}
    \]
\end{itemize}

\subsection{Estimación del error}
La estimación del error básico se calcula como:
\[
\Delta x = |x_0 - x|
\]

\newpage
\section{Medidas Directas}

\subsubsection{Caso 1: Si se realiza una sola medición}
En este caso, el error $\Delta x$ se estima como la sensibilidad del instrumento de medición. Dependiendo del tipo de instrumento, se tiene:

\begin{itemize}
    \item \textbf{Instrumento analógico:} \\
    El error es:
    \[
    \Delta x = \frac{s}{2}
    \]
    donde $s$ representa la sensibilidad del instrumento.
    
    \item \textbf{Instrumento digital:} \\
    El error se toma como:
    \[
    \Delta x = s
    \]
    donde $s$ es la sensibilidad del instrumento.
\end{itemize}

\subsubsection{Caso 2: Si se realizan varias mediciones}
Cuando se realizan múltiples mediciones, el error $\Delta x$ se estima como la desviación estándar de las mediciones realizadas, y el valor final se expresa como:
\[
x \pm \Delta x
\]
donde:
\begin{itemize}
    \item $\Delta x$ es la desviación estándar.
    \item $x$ es el promedio de las mediciones.
\end{itemize}

\newpage
\section{Medidas Indirectas}
En la propagación de errores para medidas indirectas, las fórmulas principales son:

\subsection{Suma}
Para la suma de dos valores $A$ y $B$ con sus respectivas incertidumbres $\Delta A$ y $\Delta B$, el resultado es:
\[
(A \pm \Delta A) + (B \pm \Delta B) = A + B \pm (\Delta A + \Delta B)
\]

\subsection{Resta}
Para la resta, el resultado es:
\[
(A \pm \Delta A) - (B \pm \Delta B) = A - B \pm (\Delta A + \Delta B)
\]

\subsection{Producto}
Para el producto de dos valores, la propagación de errores se calcula como:
\[
(A \pm \Delta A)(B \pm \Delta B) = A \cdot B \pm A \cdot B \left( \frac{\Delta A}{A} + \frac{\Delta B}{B} \right)
\]

\subsection{Cociente}
Para el cociente de dos valores, el resultado es:
\[
\frac{A \pm \Delta A}{B \pm \Delta B} = \frac{A}{B} \pm \frac{A}{B} \left( \frac{\Delta A}{A} + \frac{\Delta B}{B} \right)
\]

\subsection{Multiplicación por un escalar}
Si un valor $A \pm \Delta A$ se multiplica por un escalar $C$, el resultado es:
\[
C(A \pm \Delta A) = CA \pm C \Delta A
\]

\subsection{Potencia}
Si un valor $A \pm \Delta A$ se eleva a una potencia $n$, la propagación del error se calcula como:
\[
(A \pm \Delta A)^n = A^n \pm n \cdot A^{n-1} \cdot \Delta A
\]